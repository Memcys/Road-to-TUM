
\section{签证类型}
申请德国长期签证---留学签证。 
按照短期交换团组的程序申请签证,团组号由访学办老师联系德国大使馆申请,访学办老师取得团组号之后会将团组号告诉所有学生。
短期交换程序的基本流程及要求见附件4

\section{申请流程}
\begin{enumerate}
\item 登录 \url{http://videx.diplo.de/} 填写申请表并下载打印;
\item \sloppy 在德国驻华大使馆留德人员审核部 APS (\url{www.aps.org.cn/zh/}) 注册帐号。\url{https://www.aps.org.cn/zh/verfahren-und-services-deutschland/austauschverfahren}。注册时需要填写 14 位考生号。);
\item 预约递签。网址 \url{https://service2.diplo.de/rktermin/extern/choose_category.do?locationCode=peki&realmId=12&categoryId=156&request_locale=de}. 请以德文界面进入。后续表格中将有 德/英/中 三语。我以英语进入时,表格中多项标题为空白。 
\item 递签,并缴纳申请费(60 欧元,以人民币支付)。
\item 领取护照(6 周以上)。
\end{enumerate}

\section[德意志银行开户]{德意志银行开户}
开户的体验在 \ref{ap:bank} 附录中有补充完善。

本节以下内容参考\href{https://china.db.com/china/docs/Deutsche_Bank-China-Account-Opening-Process-And-Introduction.pdf}{开户流程介绍及登记卡}。

工作时间:(北京分行和上海分行) 周一至周五,上午 9:30 --- 11:30 (公共假期除外)

\begin{table*}[htbp]
\caption{学生业务专线查询(北京分行)}
\label{tb:bank-communication}
\centering
\begin{tabular}{ll}
  垂询电话:(010) 59698181 & 传真:(010) 59695710 \\
  \multicolumn{2}{l}{地址:中国北京市朝阳区建国路81号华贸中心1号写字楼27层(100025)}
\end{tabular}  
\end{table*}

\emph{重要通知}(见德意志银行网页\href{https://china.db.com/china/cn/content/5777.html}{其他信息下载})
\begin{enumerate}
\item 开户表格内容请用电脑在线填写,然后选用A4纸张,单面打印两份(其中一份您自行保留),签字部分由学生本人用黑色墨水笔签写,手工填写的表格无法受理。
\item 开户表格红色标识的区域为必填项,如打印时,页面底部有 ``UNGULTIG'' 字样的提示,表示您还有未完成的部分,请参考填写说明仔细检查。
\item \sloppy 关于允许 Java 脚本运行的提示,请根据您电脑的设置接受或直接填写,需要注意的是,最终提交的表格不能带有 ``UNGULTIG'' 字样。如果本网站提供的表格您始终无法正常打印,可以尝试从我行德国官网提供的表格链接在线填写并打印,网址如下:\\
\url{https://www.deutsche-bank.de/pfb/content/pk-konto-und-karte-international-students.html?pfb_tab=34880-34884}
\item 开户表格中纳税人信息必须填写。要求提供“资金来源证明”,请注意提前准备。
\end{enumerate}
\begin{itemize}
\item 在德意志银行官网上 \url{https://china.db.com/china/cn/content/5777.html} 下载开户表格。填写方法可以按照样表,也可以在以下网站上找到填写攻略:\\
\url{https://www.sohu.com/a/238196417_100189530} (比较全)\\
\url{https://www.sohu.com/a/218431242_507614}
\item\label{de-bank-ZJR-laptop-USB} 需要准备资金来源证明原件(中英文。办理人不能为学生本人---德国使馆将认定留学生本人无收入。需要冻结资产,欧元形式。冻结时间:最好是冻结到我们去开户的前一天或者当天---\url{http://toutiao.manqian.cn/wz_9HOPNjkAgR.html}。该证明上时间只需持续一天。如 2018 年 12 月 20 日至 2018 年 12 月 21 日---\url{http://toutiao.manqian.cn/wz_16ASBZf2o3.html?tdsourcetag=s_pctim_aiomsg}. );
\item 单面打印两份开户表格,一份自留。 ZJR 建议携带笔记本电脑和 U 盘,因可能需要随时修改;
\item 护照、身份证原件,复印件一份。复印件请以 1:1 比例清晰复印在 A4 纸上(阶梯教室打印店的质量即可);
\item 德国学校提供的大学入学通知书 A4 单面打印;
\item 登记卡。手写。下载链接: \url{https://china.db.com/china/cn/content/5777.html} 中的\href{https://china.db.com/china/docs/Deutsche_Bank-China-Account-Opening-Process-And-Introduction.pdf}{登记卡};
\item 人民币现金 850 元(一年限制题款账户) / 1200 元(半年限制题款账户)。二者均可。注意,转账时仍可能按比例扣除手续费。实际为汇款。去中国银行、工商银行均需当前银行卡,暂不接受现金汇款。
\end{itemize}
% “北京开户是在一楼登记处登记,然后上 27 楼,进去拿号排队,提交开户材料,分行审核通过后,做手续费的汇款”。“手续费包括了国际快递费和分行手续费,共计人民币 850 元(直接汇款即可)”%! link needed here
% (该段文字出处待补充)

\section{留德人员审核部 (APS) 的审核证明(一份原件)}
\begin{enumerate}
  \item 在APS官网填表(选择短期交换项目,团队号见微信群),打印签字,自贴照片(“必须提交 3 张相同的白色背景的近期证件照,照片尺寸为 \SI{45x35}{\mm}. 照片必须为正面免冠照,不能遮挡眼睛。具体要求请参看 \href{https://china.diplo.de/blob/1090226/8b25f160e56c0465aa9b5d5d19f89c4f/pdf-fotomustertafel-data.pdf}{照片示例表}。”---\href{https://china.diplo.de/cn-zh/service/visa-einreise/faq-national-visa/1434978}{长期停留常见问题--9.})
  \item 身份证,护照复印件
  \item 大学在学证明原件(周三下午 13:30--16:30)。\href{https://www.aps.org.cn/wp-content/uploads/260_merkblatt_verfahren_austausch_chn.pdf}{短期交换程序须知}---“大学在读证明内容必须包括:被哪所大学录取,院系名称,所学专业,学号,学历类型,学习起\emph{止}(但老师不允许写该“止”,只允许模板上已有的内容)时间,已读完的学期数(老师勉强允许填写),并由教务处/档案室/学籍管理办公室等校级部门盖章。“
  \item 学士成绩单原件,分学年(本校只提供不分学年的成绩单。周一下午和周三上午,办公室110)。\href{https://www.aps.org.cn/wp-content/uploads/260_merkblatt_verfahren_austausch_chn.pdf}{短期交换程序须知}---“大学成绩单必须包括大学期间所学全部课程的成绩,需要分学期开具,并由教务处/档案室/学籍管理办公室等校级部门盖章。”
  \item 大学录取花名册相关页原件(联系学校招生办开具,先打电话 88256215,然后到办公楼 123 领取。注意不同人可能在不同页上,请确认自己或他人代办该项)
  \item 中德负责人联系方式(写在一张A4纸上,中方写张老师,德方写 Dalma. 可参考 \ref{tb:contacts} 附录)
  \item 汇款单收据的复印件。每人 1000 元人民币。交由一人统一汇款。尽可能在同一天汇款与交寄材料,或稍迟几天递交材料。否则会影响到材料和审核费的对帐,影响审核进度。请注意,只有在材料齐全后才开始进行材料审核。
\end{enumerate}
材料集齐后统一交给张顶兰老师,由她交给APS.

\subsection{Videx 表格}
请参照 \href{https://www.aps.org.cn/wp-content/uploads/Beispiel-Videx.pdf}{VIDEX 模板} 填写。但该模板的 VIDEX 表格版本低于当前 VIDEX 官网版本。

以下主要来自 ZZY 发在 QQ 群里的截图:(百度(%! link needed here
链接待补充),一般性建议)
\begin{itemize}
\item 旅行费用和逗留期期间的生活费:“由申请人支付”
\item 支付方式:“现金”
\item 主要旅行事由:“留学”
\item 申请入境次数:“一次入境”
\item 预计进入申根区日期:与保险日期保持一致
\item 预计离开申根区日期:进入申根区日期 90 天后
\item 预计逗留期或过境期(天数):90
\end{itemize}
以下根据此次填写经历提取的注意项,由 LMY 添加。
\begin{itemize}
  \item 出生日期格式 mm.dd.yyyy (目前按提示填写 dd.mm.yyyy 则报错)
  \item 中文界面填写并生成的 VIDEX 表格只有 6 页。英文界面填写并生成的 VIDEX 表格有 7 页。
  \item 仅需打印尾页(含二维码那一页)
  \item 尾页有 1 个一维码, 3 个二维码(英文界面填写。如果是中文界面填写,可能为 1 个一维码, 2 个二维码)
\end{itemize}

\section{关于递签}\label{sec:visa}
\sloppy 预约递签的网址: \href{https://service2.diplo.de/rktermin/extern/choose_category.do?locationCode=peki&realmId=12&categoryId=156&request_locale=de}{在线系统预约签证受理}---见 \url{https://www.aps.org.cn/zh/verfahren-und-services-deutschland/visum-fur-deutschland} (本节内容均参考该网页)中“短期交换申请人(A程序)”。
\begin{description}
\item[最早可于行程开始前三个月申请签证] ---见 \url{https://china.diplo.de/cn-zh/service/visa-einreise/faq-schengenvisa/1434980} 问题 8.
\item[重新预约] ---以下两条见\sloppy \url{https://www.aps.org.cn/zh/verfahren-und-services-deutschland/visum-fur-deutschland} 中“\textbf{关于签证预约}”。预约提交后会收到系统发送的确认邮件,如果预约了错误日期或者预留的信息有误,请点击该邮件中的取消链接,然后重新预约即可
\item[重复预约被取消的情形] 如果申请人改动个人信息(比如护照号码,姓名,电话号码)进行重复预约,所有预约将被系统取消并且不会告知申请人! 
\item[重新预约递签的说明(仅限在北京审核部递签的情况)] ---见 \url{https://www.aps.org.cn/zh/verfahren-und-services-deutschland/visum-fur-deutschland}\\
通过在线系统预约北京审核部递签的申请人(C 程序 D 程序 A 程序),重新预约时有以下几种情况:
\begin{itemize}
\item 如果在递签日之前要取消预约,点击预约确认邮件中的取消链接,即可重新预约。
\item 如果递签日没有成功递签,
\begin{itemize}
  \item 若护照号码更换,请使用新号码直接预约即可
  \item 若护照号码不变,请先用英文或德文发送邮件至 \href{mailto:visa@peki.diplo.de}{visa@peki.diplo.de}
\end{itemize}
\end{itemize}
邮件需提供个人信息,申请取消上次预约(不会收到回复邮件),在发出邮件 24 小时后可重新预约。
\item[护照回寄服务费 (EMS)]

德国使馆通过中国邮政的特快专递服务 (EMS) 将护照回寄给签证申请人。此形式仅限于北京辖区、上海辖区。

北京辖区的签证申请人请按如下方式支付快递费用:
中国邮政集团公司将安排工作人员代收 \underline{\label{inline-money}\textbf{33 元}快递服务费},地点在 DRC 外交办公大楼1层东南角(北京银行旁)。申请人交纳现金后,会得到一张支付凭证,\underline{\textbf{递签时请出示该凭证}}。

递签之后申请人会得到一张快递单回执。请妥善收存此回执,按照上面的快递单号申请人可以查询护照寄出与否。
\item[有关联系方式及代办人的附加证明\label{inline-additional-proof}] ---见 \url{https://www.aps.org.cn/zh/verfahren-und-services-deutschland/visum-fur-deutschland}.
签证材料中需要附上一份\href{https://china.diplo.de/blob/1341728/895a5533a3c35c4fd2fbc21e92d6dfa3/pdf-formular-zusatzerklaerung-erreichbarkeit-data.pdf}{有关联系方式及代办人的附加证明}。\\
填写时请注意:
\begin{itemize}
  \item 受理号不填写
  \item 个人信息中文填写
  \item 选择项前面打“X”
  \item 本人递签不填写代办信息
  \item 填写日期和中文签名(必须本人亲笔签名)
\end{itemize}
\item[短期交换申请人(A程序)]
在北京审核部递签的北京广州成都沈阳四个辖区的申请人必须通过\href{https://service2.diplo.de/rktermin/extern/choose_category.do?locationCode=peki&realmId=12&categoryId=156&request_locale=de}{在线系统预约签证受理} 预约签证受理。注意:\underline{只能预约周四和/或周五}。在签证材料准备完整的情况下预约一个递签时间,按照预约时间由本人亲自在北京审核部递交材料。北京辖区的申请人可以在审核部采集指纹。

如果签证申请人计划在德国居留不超过一年,请在签证材料中附上一份\href{https://www.aps.org.cn/wp-content/uploads/Belehrung_KV.pdf}{保险声明}。

注意:请务必保证预约信息和申请人信息的一致性并且只能在申请人所属程序相对应的递签日进行预约!如果预约信息不完整或者和申请人信息不一致(例如护照号码,身份证号码,出生日期等等)或者预约日期不是申请人所属程序相对应的递签日,该签证申请不会被受理,直接取消当日的递签资格!
\end{description}

% \section{所需材料}

\section{递签所需材料(如无说明,一份原件加两份复印件)}\label{sec:visa-material}
请注意,我们在递签时发现,APS 递签处给出的要求不同于官网的要求。请以递签处要求为准。

\subsection{2019 年 1 月 17 日递签材料}\label{sec:visa-material-17Jan}
本更新内容在附录 \ref{ap:visa-figures} 中有补充。

先出示护照,随后递交材料原件,最后递交两套复印件。本部分见图 \ref{fig:order-1}, \ref{fig:order-2} 和 \ref{fig:APS-forms}(拍照)。
\begin{description}
  \item[原件]
  \begin{enumerate}
    \item 护照(首页用曲别针夹一张白底二寸照片,不能遮挡耳朵和眉毛, 不能过度修图) 
    \item EMS 邮寄单、VIDEX 二维码录取证明、资金证明、在读证明、保险证明、审核证书、联系人说明
  \end{enumerate} 
  \item[复印件] 分两套。每套材料由上至下按如下顺序放置。\textbf{\color{red}顺序不对不予受理}。 
  \begin{enumerate}
    \item 签证申请表 (首页右上角粘贴白底二寸照片, 地址栏为现居地址, 末页中文签名) 

    注意:申请表建议德文填写且不能有涂改痕迹!(ZZY 划掉了拼音姓名,也让过了) 
    
    【签证申请表填写细节与学长的模板不太一致!!!】 
    
    \item 签证申请补充(附加)声明(中文签名) 
    
    \item 护照首页复印件 
    
    \item 德方录取证明复印件:高校录取通知书(黑白、彩打均可)
    
    \item 经济来源证明复印件 
    
    \item 语言水平证明复印件(中文证明必须附德文翻译) 
    
    \item 在读证明复印件,并附德文翻译 
    
    \item 德文个人简历(至今为止的全部经历,必须有本人联系电话) 
    
    \item 德文留学动机说明 
    
    \item 审核证书复印件 
    
    \item 医疗保险证明复印件 
    
    \item 保险声明(如果在德居留不超过一年,拼音签名+中文签名,入境出境日期和签证申请表保持一致)
  \end{enumerate} 
\end{description} 

\section{递签备注}
\begin{itemize}
\item 缺少材料,将被拒签
\item 除了 VIDEX 二维码、照片、护照和 EMS 快递单,其他原件会退回
\item\label{visa-EMS} 对于 EMS 快递单,图 \ref{fig:order-1} 中未要求。但递签成功后必须提交该快递单。故请一并携带。
\item 大楼 B2 层有打印店,\textbf{黑白打印 10 元/张}。%,以及还不准在等待室用手机和电脑,所以最好提前都备好材料。 
\item 签证申请表中
\begin{enumerate}
  \item “路名和号码(如知晓)”必填,可填 TUM 提供的宿舍或学校的相关信息
  \item “是否保留德国以外的常住地?” 勾选 ``ja 是''
  \item “如是,在何地?”可填写 ``Peking, China'' 或 ``Beijing, China''等, 把 ``China'' 写成 ``P R China'' 应该也行。似乎有人填的国科大学校街道,也被允许了
\end{enumerate}
\item 所有材料尾页的签名均要求手写
\item 签证申请表应该也允许\textbf{手写}
\item 签名对语言有要求的,即使填写了要求外的语言,也不必划掉,在其周围找地方用指定的语言填写即可
\item 附加声明中 ``Barcode/受理号'' 不填
\item 审核人和保安均为中国人,且都很耐心
\item VIDEX 二维码有两个(英文界面填写)或三个(中文界面填写),均被认可
\item 从地铁 B 口(见图 \ref{fig:real} 出来,左前方(北偏西)有一幢写着 APS 的低矮建筑。向右前方(东偏北方向)行进,有一个下沉广场。大楼东侧有北京银行。紧贴北京银行的南侧,有一较不起眼的通道。刚入通道,有两扇合金/木包裹的玻璃门(图 \ref{fig:route} 中标注 ``EMS'' 的地方),门上贴着 ……EMS…… 小条。如未发现,可继续深入通道,里面有保安,可询问。
\item APS 的大字牌面北立在大楼前。
\item APS 递签处禁止使用手机等电子设备。(当然,我们拍照属于违规操作。)
\end{itemize}