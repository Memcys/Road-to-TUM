
\section{校园外}
\begin{itemize}
  \item 在市政厅办户口(离开德国前注销户口、保险、银行卡,退房)
  \item 银行卡解冻
  \item 地铁卡买全票(90 欧元。我们的仅为 ``Basic Semester Ticket''.)
  \item 保险补全信息
  \item 注意某些课程可能会提前开课(比如物理系的 GPU 计算课程,大约在 4 月 11 日开始首堂课,并表示如果错过首堂课,将分配不到期末任务。)
  \item (From the airport to) {\color{blue}Garching}: either with S1 to Neufahrn, from there take bus 690; or S-Bahn S8 to Ismaning and take bus 230. (from \href{https://distributed-campus.org/tumwelcomeguide/portal/media-type/html/language/en/role/student/page/default.psml/js_panename/DCContentObjectAusgabe/language/en/idgroup/19/id/96/typ/text}{TUM Welcome Guide})
  \item If you actually want to go to {\color{blue}Munich city center}, look for the white "S" in a green circle, which is the sign for the suburban rail station located in the underground level of the airport. 2 suburban train (S-Bahn) lines go to the airport, {\color{blue}S1 and S8}. If you want to go to the Central Train Station (Hauptbahnhof), you can take either one.
  \item Since the airport is located on the edge of the city, it is a good idea to buy the {\color{blue}Airport-City-Day-Ticket} for the entire network. This way, you can ride all of the public transportation provided by MVV (buses, S-Bahn, U-Bahn, and tram) until the next morning 6 am, without having to buy another ticket. Ticket machines are located outside the escalator leading to the platform, and an Airport-City-Day-Ticket is \euro{13,00}. It is a good idea to have cash on hand for buying your ticket. After you've bought your ticket, insert it in the blue, waist high validation machines also located on the entrance to the platform. This stamps the time and date, which tells you when you started to use the ticket (and how long it is valid for). There is no barrier to enter the train or platform, so just get on the train with your validated ticket. Remember not to lose your ticket, because a conductor may come by to see if you have bought one. (Riding without a ticket is called "Schwarzfahren" and will lead to a \euro{60} fine!)
  \item If you only go to Garching or Weihenstephan (Freising) the ticket will be cheaper.
  \item \href{https://distributed-campus.org/tumwelcomeguide/portal/media-type/html/language/en/role/student/page/default.psml/js_panename/DCContentObjectAusgabe/language/en/idgroup/19/id/111/typ/text}{Cell Phones (called "Handy" in Germany)}
\end{itemize}

\section{TUM}
本节完全引自 email ``Wichtige Informationen für Ihr Austauschstudium an der TUM / important information for your exchange at TUM'' 中附件。
% \attachfile{../Attached/Information_SoSe19_engl.pdf}。

\subsection{TUM Student Card}
Your TUM Student Card needs to be picked up: \\
When: April 1st, 2019 -- April 18 th , 2019: Please respect these dates!* \\
Time: Monday -- Thursday: 9 a.m. -- 12 p.m. and 1 p.m. -- 4 p.m., Friday: 9 a.m. -- 12 p.m. \\
Where: Room 0120, Arcisstrasse 21, Munich \\
What do I need to bring: Your passport / National ID Card.

Please bear in mind that due to the information meeting on April 9th 2019 (see next point) you can only collect your Student Card from 9 a.m. -- 12 a.m. on that day!

Note that April 19 th – 22 nd is the Easter break: offices and TUM are closed.


\subsection{Information meeting for your time at TUM and in Munich}
When: Tuesday, April 9 th , 2019 \\
Time: 2 p.m. -- 3 p.m. \\
Where: Room 2750, Karl Max von Bauernfeind Hörsaal \\
This meeting is organized by the Incoming Office of the TUM International Center. We will be available to
answer any special individual questions after the meeting. \\
Additionally, many TUM departments offer their own information meetings to inform you about topics that
are relevant for the department (e.g. courses, exams, etc.). In this case, your TUM department will inform
you about the place and time. Please attend both information meetings!
Please bear in mind that due to this meeting you can only collect your Student Card from
9 a.m. -- 12 a.m. on that day!


\subsection{Orientation weeks}
When: April 8 th , 2019 – April 22 nd , 2019 \\
Program (will be updated in March): \href{http://www.international.tum.de/en/coming-to-tum/tumi/}{Program} \\
Contact: \href{mailto:tumi@zv.tum.de}{tumi@zv.tum.de}(organized by TUMi, Mentoring program for exchange students) \\
During orientation weeks different activities will be offered to get to know TUM, Munich and other students. \\
Participation is not obligatory, but recommended! You can take part in a single or multiple activities. \\
Many activities are free of charge (e.g. campus tours), for others you need to pay a small amount (e.g.
entry-, travelling fees).


\subsection{Location of your TUM department / Contacts in your TUM department}
Munich: Electrical and Computer Engineering, Civil Geo and Environmental Engineering, Architecture,
School of Management, Sport and Health Sciences, Medicine, School of Education, School of Governance \\
Garching: Mechanical Engineering, Chemistry, Physics, Informatics, Mathematics, Munich School of
Engineering. \\
Freising: School of Life Sciences Weihenstephan



\subsection{Language courses}
The TUM Language Center offers German and other language courses. Unfortunately, no guarantee
can be given for places in TUM German or other language courses!
Registering for language courses
\begin{itemize}
  \item First, you need to fill out a so-called placement test via Moodle.
  \item Afterwards, you can register for courses and exams via TUMonline or Moodle.
\end{itemize}


\subsection{Legal requirements / visa}
\begin{enumerate}
  \item Once you have found a room or apartment and moved in, you must \href{https://www.muenchen.de/rathaus/home_en/Department-of-Public-Order/Registration-Deregistration}{register at your local resident
  registration office} within two weeks.
  \item If you need to apply for a (temporary) residence permit in Germany (see link below), you need to do so
  after your arrival in Germany.
\end{enumerate}
All information (register, permit, if you need a visa) can be found here under “\href{http://www.international.tum.de/en/exchangestudents/}{Legal Requirements}”.


\subsection{Open a bank account}
Since 2018 it is obligatory to show a tax-ID for opening a bank account.
\begin{itemize}
  \item When coming to Germany, please bring with you your own tax-ID from your home country!
  \item Where to find it? See \href{http://www.oecd.org/tax/automatic-exchange/crs-implementation-and-assistance/tax-identification-numbers/}{here} or ask your home university / family.
\end{itemize}
You can also get a German tax-ID:
\begin{itemize}
  \item Find a room in the city / environment of Munich.
  \item Register at your local resident registration office (see former point).
  \item Around 6-8 weeks later you will receive it via post.
  
\end{itemize}
Important: Be aware that 6-8 weeks are often too late for opening a bank account, as e.g. landlords, TUM
sports (if taking part) need your bank data earlier. So it ́s better to use the one from your home country.

\subsection{A summary: Your contacts at TUM}
\begin{itemize}
  \item Courses, examinations, Learning Agreement, all signatures: \href{http://www.international.tum.de/en/internationalaffairs/}{TUM's International Affairs Delegates}
  \item Language courses: \href{http://www.sprachenzentrum.tum.de/en/languages/german-as-a-foreign-language/}{TUM Language Center}
  \item Accommodation: Anna Kondratskaya (\href{mailto:incoming\_help@zv.tum.de}{incoming\_help@zv.tum.de})
  \item Visa: the German Embassy in your home country
  \item All other questions: until end of February: Dalma Alagha (\href{mailto:alagha@zv.tum.de}{alagha@zv.tum.de}) \\
  From 1 st of March: \href{mailto:incoming@zv.tum.de}{incoming@zv.tum.de}
\end{itemize}

These aforementioned points and other information like sports at TUM, opening a bank account, etc. can
also be found here: “\href{http://www.international.tum.de/en/welcome-to-tum/international-exchange-students/exchange-students/}{Useful Information about living in Munich}”