\documentclass{article}
\usepackage{ctex}
\usepackage{enumerate}
\usepackage{enumitem}   % [inline] option for inline enumerate, following https://tex.stackexchange.com/a/146311
\usepackage[colorlinks]{hyperref}
\usepackage{bookmark}
\usepackage[left=3cm,right=3cm]{geometry}
\usepackage{multicol}
\usepackage[official]{eurosym}
\usepackage{booktabs}
\usepackage{xcolor}
\usepackage{siunitx}

% Define my "abstract" title, following https://en.wikibooks.org/wiki/LaTeX/Document_Structure#Abstract
\renewcommand{\abstractname}{声明}

\title{Road to TUM\\for UCASer in Summer 2019}
\author{UCASer 16}
\date{\today}

\begin{document}
\maketitle

\vfill
\abstract{本文所参考内容将尽可能给出出处,且本文骨干内容主要基于:
\begin{enumerate}          % inline enumerate with *, given [inline] option in package enumitem
  \item 16级 QQ 群文件 “\href{https://docs.qq.com/doc/DSHd2dlFVZXpodEpq}{TUM 流程}”,
  \item 张顶兰老师邮件的附件,
  \item TUM Dalma 邮件及其附件。
\end{enumerate}
}

\vfill

这份文档,既为了方便此次访学,也是希望能够为往后的 TUM 交流生铺一条路。

这篇文档由 \LaTeX 写作编译成 pdf. 目前放在 \href{https://www.overlear.com}{overleaf.com} (项目名为 \href{https://www.overleaf.com/2269426218fxwmgyxjywnn}{Road-to-TUM}) 以及 \href{https://github.com}{github.com} (项目名为 \href{https://github.com/Memcys/Road-to-TUM.git}{Road-to-TUM}, 同时有利用 Pandoc 从 .tex 转换为 .docx 的 Word 文档)上。目前 overleaf 网络访问不畅通,故新添了 github 项目。

主要内容的增补将仍在文件“\href{https://docs.qq.com/doc/DSHd2dlFVZXpodEpq}{TUM 流程}”中完成。希望有兴趣的同学可以参与这个文档编写项目。

目前文档正在增添和修补中。文中“做什么”“建议做什么”“为什么这么做”等杂糅。有些地方指出 见附件 等,是直接粘贴的原句。但附件没有上传。此外,videx, 递签等部分细节仍待补充。各节内容都需补充。中英交替,语言表述不够准确、精炼等情况,暂不处理。

\newpage

\section{To-Do List}
(使用合适的 pdf 阅读器,可以点选/取消可选框。)
\begin{Form}
\small
% Following https://tex.stackexchange.com/a/310574
  \begin{enumerate}[label = {\CheckBox[width=.1in,height=.1in,name=process\theenumi]{}}]
  \item 德国保险
    \begin{multicols}{2}
    \begin{enumerate}[label = {\CheckBox[width=.1in,height=.1in,name=subprocess\theenumii]{}} \arabic*.]
      \item (AOK)申请表
      \item certificate for enrollment
      \item email \href{mailto:alagha@zv.tum.de}{alagha@zv.tum.de}
    \end{enumerate}  
    \end{multicols}
  \item 签证申请
  % \begin{multicols}{3}
  \begin{enumerate}[label = {\CheckBox[width=.1in,height=.1in,name=subprocess\theenumii]{}}]
    \item videx form
    \item APS 材料
      \begin{multicols}{2}
      \begin{enumerate}[label = {\CheckBox[width=.1in,height=.1in,name=subsubprocess\theenumiii]{}} \arabic*.]
        \item APS 注册
        \item 身份证,护照复印件
        \item 在学证明原件
        \item 成绩单原件
        \item 录取花名册相关页原件
        \item 中德负责人联系方式
        \item 汇款单收据的复印件
      \end{enumerate}
      \end{multicols}
    \item 签证材料
      \begin{multicols}{2}
      \begin{enumerate}[label = {\CheckBox[width=.1in,height=.1in,name=subsubprocess\theenumiii]{}} \arabic*.]
      \item 1 份 videx 表格最后页(第 7 页)
      \item 2 份德文居留许可申请表及 \\
      3 张护照照片
      \item 护照(原件)及 2 份护照照片页复印件
      \item TUM 通知书及语言水平证明
      \item 存款证明
      \item 英语在学证明
      \item 个人简历
      \item 留学德国的理由说明书
      \item 医疗保险证明
      \end{enumerate}
      \end{multicols}
    \item 德意志银行开户
      \begin{multicols}{2}
      \begin{enumerate}[label = {\CheckBox[width=.1in,height=.1in,name=subsubprocess\theenumiii]{}} \arabic*.]
        \item 开户表格(2 份,单面)
        \item 资金来源证明(中英文)
        \item 护照、身份证原件 + 1 份复印件
        \item TUM 通知书 A4 单面
        \item 登记卡(手写)
        \item 人民币现金 850 元(实际为汇款)
      \end{enumerate}
      \end{multicols}
    \item APS 预约递签
    \item APS 递签
    \item 领取护照
    \item APS 的审核证明/审核证书/传真
  \end{enumerate}
  % \end{multicols}
  \item 德国居留许可申请
  \item 住宿
  \item 航班
  \item Registration for TUM
  \begin{multicols}{2}
  \begin{enumerate}[label = {\CheckBox[width=.1in,height=.1in,name=subprocess\theenumii]{}}]
    \item \euro{129,40}
    \item health insurance
    \item up-load a photo in TUMonline (and change password)
  \end{enumerate}
  \end{multicols}
  \end{enumerate}
\end{Form}

\newpage
\tableofcontents
\newpage

以下各节间安排顺序与实际操作顺序无直接关系。

\section{德国保险}
到德国留学的学生,每人均需购买一份医疗保险才能成功申请签证。%有关医疗保险的细节如下:
\begin{itemize}
\item 可选择购买保险的公司及相应价格如下表:%! figure or table needed here
TUM 给的建议是选择 AOK 公司。
\item 保险购买步骤
\begin{enumerate}
\item 将本人姓名、出生日期、国籍、性别(Name,Date of birth,Home country,Sex)等基本信息写邮件至 \sloppy \href{mailto:muenchen.student@service.by.aok.de}{muenchen.student@service.by.aok.de};% Insert email address. Following https://tex.stackexchange.com/a/276.
\item 等对方回邮件填写申请表并签名,TUM官网有 AOK 申请表(见附件2),仅供参考,请以上述邮箱发过来的表格为准;
\item 申请表填好后,打印 application 并填写,扫描后 ``together with a copy of your passport and admission letter of the university'' 发邮件回去。%将扫描件发至给大家发申请表的邮箱地址,
通常一天之后大家就能收到入学所需的保险证明(certificate for enrollment);
% \item 拿到保险公司的保险证明(一个名为 Versicherungsbescheinigung 的文件),将保险证明通过电子邮件发给TUM的负责人Dalma Alagha(\href{mailto:alagha@zv.tum.de}{alagha@zv.tum.de}) (by mentioning your name and your TUM regisration number ``Matrikelnummer''. 见 )
\end{enumerate}
\item TUM保险购买的详细英文指南,以及关于保险的各项细节见附件3;
\end{itemize}

\section{签证申请}
\subsection{签证类型}
申请德国长期签证---留学签证。 
按照短期交换团组的程序申请签证,团组号由访学办老师联系德国大使馆申请,访学办老师取得团组号之后会将团组号告诉所有学生。
短期交换程序的基本流程及要求见附件4

\subsection{申请流程}
\begin{enumerate}
\item 登录 \url{http://videx.diplo.de/} 填写申请表并下载打印;
\item \sloppy 在德国驻华大使馆留德人员审核部 APS (\url{www.aps.org.cn/zh/}) 注册帐号。\url{https://www.aps.org.cn/zh/verfahren-und-services-deutschland/austauschverfahren}。注册时需要填写14位考生号。);
\item 预约递签。网址 \url{https://service2.diplo.de/rktermin/extern/choose_category.do?locationCode=peki&realmId=12&categoryId=156&request_locale=en}. 
\item 递签,并缴纳申请费(60欧元,以人民币支付)。
\item 领取护照(6周以上)。
\end{enumerate}
注:
\begin{description}
\item[最早可于行程开始前三个月申请签证] ---见 \url{https://china.diplo.de/cn-zh/service/visa-einreise/faq-schengenvisa/1434980} 问题 8.
\item[重新预约] ---以下两条见\sloppy \url{https://www.aps.org.cn/zh/verfahren-und-services-deutschland/visum-fur-deutschland} 中“\textbf{关于签证预约}”。预约提交后会收到系统发送的确认邮件,如果预约了错误日期或者预留的信息有误,请点击该邮件中的取消链接,然后重新预约即可
\item[重复预约被取消的情形] 如果申请人改动个人信息(比如护照号码,姓名,电话号码)进行重复预约,所有预约将被系统取消并且不会告知申请人! 
\item[重新预约递签的说明(仅限在北京审核部递签的情况)] ---见\sloppy \url{https://www.aps.org.cn/zh/verfahren-und-services-deutschland/visum-fur-deutschland}\\
通过在线系统预约北京审核部递签的申请人(C程序D程序A程序),重新预约时有以下几种情况:
\begin{itemize}
\item 如果在递签日之前要取消预约,点击预约确认邮件中的取消链接,即可重新预约。
\item 如果递签日没有成功递签,
\begin{itemize}
  \item 若护照号码更换,请使用新号码直接预约即可
  \item 若护照号码不变,请先用英文或德文发送邮件至visa@peki.diplo.de
\end{itemize}
\end{itemize}
邮件需提供个人信息,申请取消上次预约(不会收到回复邮件),在发出邮件24小时后可重新预约。
\item[有关联系方式及代办人的附加证明] ---见 \url{https://www.aps.org.cn/zh/verfahren-und-services-deutschland/visum-fur-deutschland}.
签证材料中需要附上一份有关联系方式及代办人的附加证明。\\
填写时请注意:
\begin{itemize}
  \item 受理号不填写
  \item 个人信息中文填写
  \item 选择项前面打“X”
  \item 本人递签不填写代办信息
  \item 填写日期和中文签名(必须本人亲笔签名)
\end{itemize}
\end{description}

\subsection{所需材料}
\subsubsection{留德人员审核部 (APS) 的审核证明(一份原件)}
\begin{enumerate}
  \item 在APS官网填表(选择短期交换项目,团队号见微信群),打印签字,自贴照片(“必须提交 3 张相同的白色背景的近期证件照,照片尺寸为 \SI{45x35}{\mm}. 照片必须为正面免冠照,不能遮挡眼睛。具体要求请参看 \href{https://china.diplo.de/blob/1090226/8b25f160e56c0465aa9b5d5d19f89c4f/pdf-fotomustertafel-data.pdf}{照片示例表}。”---\href{https://china.diplo.de/cn-zh/service/visa-einreise/faq-national-visa/1434978}{长期停留常见问题--9.})
  \item 身份证,护照复印件
  \item 大学在学证明原件(周三下午 13:30--16:30)。\href{https://www.aps.org.cn/wp-content/uploads/260_merkblatt_verfahren_austausch_chn.pdf}{短期交换程序须知}---“大学在读证明内容必须包括:被哪所大学录取,院系名称,所学专业,学号,学历类型,学习起\emph{止}(但老师不允许写该“止”,只允许模板上已有的内容)时间,已读完的学期数(老师勉强允许填写),并由教务处/档案室/学籍管理办公室等校级部门盖章。“
  \item 学士成绩单原件,分学年(本校只提供不分学年的成绩单。周一下午和周三上午,办公室110)。\href{https://www.aps.org.cn/wp-content/uploads/260_merkblatt_verfahren_austausch_chn.pdf}{短期交换程序须知}---“大学成绩单必须包括大学期间所学全部课程的成绩,需要分学期开具,并由教务处/档案室/学籍管理办公室等校级部门盖章。”
  \item 大学录取花名册相关页原件(联系学校招生办开具,先打电话 88256215,然后到办公楼 123 领取。注意不同人可能在不同页上,请确认自己或他人代办该项)
  \item 中德负责人联系方式(写在一张A4纸上,中方写张老师,德方写 Dalma. 可参考 \ref{ap} 附录)
  \item 汇款单收据的复印件。每人 1000 元人民币。交由一人统一汇款。尽可能在同一天汇款与交寄材料,或稍迟几天递交材料。否则会影响到材料和审核费的对帐,影响审核进度。请注意,只有在材料齐全后才开始进行材料审核。
\end{enumerate}
材料集齐后统一交给张顶兰老师,由她交给APS.

\subsubsection{签证所需材料(如无说明,一份原件加两份复印件)}
参考 \url{https://china.diplo.de/cn-zh/service/visa-einreise/nationales-visum/1345434#content_2} 中文件\href{https://china.diplo.de/blob/1341652/6085aece6f28eb2e16b4d851c3632005/pdf-merkblatt-natvisum-studium-data.pdf}{留学签证须知}。如没有另外说明,其余所递交的材料一式三份(原件加两份复印件),即需要递交两套(?)相同的材料。每套材料按下列顺序排列:
\begin{enumerate}
\item 1 份已生成 PDF 文件的 videx \url{https://videx.d
iplo.de/} 表格中的最后页(第 7 页)的打印件
\item 2 份用德文完整填写并亲笔签名的居留许可\underline{申请表}%以及符合外国人居留法第53和54条要求的声明(可从 \url{https://china.diplo.de/cn-zh/service/visa-einreise/nationales-visum/1345434#content_2} 下载。
长期签证申请表(德文中文) \url{https://china.diplo.de/blob/1427944/d2799d8c6533d3bf53d4a936bcb4a8e8/pdf-antrag-natvisum-data.pdf} 及 3 张相同的白色背景的近期\underline{护照照片}\\
(用于签证申请的护照照片必须是近期的,超过6个月的照片将不被接受。尺寸为4.5厘米 x 3.5 厘米,高分辨率和白色背景。含下巴和脖子的正面照,脸部和眼睛不得遮挡,脸部大小必须占照片的70-80%。
如证件照明显超过6个月或不符合规定,签证申请将不予受理并被退回。---见 \url{https://china.diplo.de/cn-zh/service/visa-einreise/faq-schengenvisa/1434980} 问题 21);
\item 亲笔签名的\underline{旅行护照},并附上护照照片页复印件 2 份;护照有效期应超出签证有效期 3 个月以上;
\item 德国高校\underline{录取通知书}以及录取通知书上所要求的语言水平证明(文件\href{https://china.diplo.de/blob/1341652/6085aece6f28eb2e16b4d851c3632005/pdf-merkblatt-natvisum-studium-data.pdf}{留学签证须知}指出,“必要时附上德文翻译”);
\item \sloppy 为期一年至少 8640 欧元的银行存款证明(限制提款账户),(可在德国大使馆官网上找到代开德国境内存款证明的服务\url{http://www.china.diplo.de/Vertretung/china/zh/02-visa/02-nationale-visa/Einrichtung_20Sperrkonto-17022017.html});
\item \underline{在学证明},并附\textbf{德文译文}。(在本科教育网-学籍学工栏目下载\href{http://bkjy.ucas.ac.cn/index.php/jyjx/download/3619-2017-08-31-08-51-16}{在学证明申请表及模板},填写完本人信息之后,持申请书到本科部找李涵老师盖章,然后到学生处办理在学证明盖章手续;请大家填好申请表和在学证明模板之后,统一交给一个人办理);
\item 截至申请递交前的
完整无间隙的\underline{表格式德文个人简历}
\item \textbf{德文版}的\underline{留学理由说明书},详细叙述留学的原因和留学对将来的职业生涯的影响。
\item \underline{医疗保险证明}:自入境德国起至少 90天有效
\item 德国户籍登记机构的注销户籍登记证明(只针对此前在德国常住过,并持有有效长期签证/居留证或长期签证、居留证过期不满六个月的)
\item APS 的\underline{审核证明/审核证书/传真}
\end{enumerate}

\subsubsection{Videx 表格}
请参照 \href{https://www.aps.org.cn/wp-content/uploads/Beispiel-Videx.pdf}{VIDEX 模板} 填写。但该模板的 VIDEX 表格版本低于当前 VIDEX 官网版本。

以下主要来自张兆昱发在 QQ 群里的截图:(百度(%! link needed here
链接待补充),一般性建议)
\begin{itemize}
\item 旅行费用和逗留期期间的生活费:“由申请人支付”
\item 支付方式:“现金”
\item 主要旅行事由:“留学”
\item 申请入境次数:“一次入境”
\item 预计进入申根区日期:与保险日期保持一致
\item 预计离开申根区日期:进入申根区日期 90 天后
\item 预计逗留期或过境期(天数):90
\item 出生日期格式 mm.dd.yyyy (目前按提示填写 dd.mm.yyyy 则报错。李明阳添加)
\end{itemize}
注:
\begin{itemize}
  \item 中文界面填写并生成的 VIDEX 表格只有 6 页。英文界面填写并生成的 VIDEX 表格有 7 页。
  \item 仅需打印尾页(含二维码那一页)
  \item 尾页有 1 个一维码, 3 个二维码
\end{itemize}

\subsubsection[德意志银行开户]{德意志银行开户}
本节参考\href{https://china.db.com/china/docs/Deutsche_Bank-China-Account-Opening-Process-And-Introduction.pdf}{开户流程介绍及登记卡}。

工作时间:(北京分行和上海分行) 周一至周五,上午 9:30 --- 11:30 (公共假期除外)

\begin{table*}[htbp]
\caption{学生业务专线查询(北京分行)}
\label{tb:bank-communication}
\centering
\begin{tabular}{ll}
  垂询电话:(010) 59698181 & 传真:(010) 59695710 \\
  \multicolumn{2}{l}{地址:中国北京市朝阳区建国路81号华贸中心1号写字楼27层(100025)}
\end{tabular}  
\end{table*}

\emph{重要通知}(见德意志银行网页\href{https://china.db.com/china/cn/content/5777.html}{其他信息下载})
\begin{enumerate}
\item 开户表格内容请用电脑在线填写,然后选用A4纸张,单面打印两份(其中一份您自行保留),签字部分由学生本人用黑色墨水笔签写,手工填写的表格无法受理。
\item 开户表格红色标识的区域为必填项,如打印时,页面底部有 ``UNGULTIG'' 字样的提示,表示您还有未完成的部分,请参考填写说明仔细检查。
\item \sloppy 关于允许 Java 脚本运行的提示,请根据您电脑的设置接受或直接填写,需要注意的是,最终提交的表格不能带有 ``UNGULTIG'' 字样。如果本网站提供的表格您始终无法正常打印,可以尝试从我行德国官网提供的表格链接在线填写并打印,网址如下:\\
\url{https://www.deutsche-bank.de/pfb/content/pk-konto-und-karte-international-students.html?pfb_tab=34880-34884}
\item 开户表格中纳税人信息必须填写。要求提供“资金来源证明”,请注意提前准备。
\end{enumerate}
\begin{itemize}
\item 在德意志银行官网上 \url{https://china.db.com/china/cn/content/5777.html} 下载开户表格。填写方法可以按照样表,也可以在以下网站上找到填写攻略:\\
\url{https://www.sohu.com/a/238196417_100189530} (比较全)\\
\url{https://www.sohu.com/a/218431242_507614}
\item 需要准备资金来源证明原件(中英文。办理人不能为学生本人---德国使馆将认定留学生本人无收入。需要冻结资产,欧元形式。冻结时间:最好是冻结到我们去开户的前一天或者当天---\url{http://toutiao.manqian.cn/wz_9HOPNjkAgR.html}。该证明上时间只需持续一天。如 2018 年 12 月 20 日至 2018 年 12 月 21 日---\url{http://toutiao.manqian.cn/wz_16ASBZf2o3.html?tdsourcetag=s_pctim_aiomsg}. );
\item 单面打印两份开户表格,一份自留。赵家瑞建议携带笔记本电脑和 U 盘,因可能需要随时修改;
\item 护照、身份证原件,复印件一份。复印件请以 1:1 比例清晰复印在 A4 纸上;
\item 德国学校提供的大学入学通知书A4单面打印;
\item 登记卡。手写。下载链接: \url{https://china.db.com/china/cn/content/5777.html} 中的\href{https://china.db.com/china/docs/Deutsche_Bank-China-Account-Opening-Process-And-Introduction.pdf}{登记卡};
\item 人民币现金 850 元。实际为汇款。去中国银行、工商银行均需当前银行卡,暂不接受现金汇款。
\end{itemize}
“北京开户是在一楼登记处登记,然后上 27 楼,进去拿号排队,提交开户材料,分行审核通过后,做手续费的汇款”。“手续费包括了国际快递费和分行手续费,共计人民币 850 元(直接汇款即可)”%! link needed here
(该段文字出处待补充)

\section{德国居留许可申请}
学生到达德国之后,停留时间超过3个月且签证有效期不足以覆盖整个访学期限的,须申请德国居留许可(如果签证有效期能够覆盖整个访学期间,则无需申请居留许可)。
\begin{itemize}
\item 申请居留许可所需文件:
\begin{enumerate}
\item 一份填写完整的网上申请表;
\item 有效护照
\item TUM 录取信
\item 医疗保险证明
\item 资金来源证明(德国境内任何一家银行开具的,每月720欧元,至少一年的限制提款证明)
\item 在慕尼黑登记注册的证明(学生到达慕尼黑后应在一星期内到 Department of Public Order (Kreisverwaltungsreferat)下属的 residence registration office 登记,需带护照、录取信、保险证明银行存款证明等文件);
\item 留学签证
\item 近期护照尺寸证件照一张
\end{enumerate}
\item 费用\\
申请德国居留许可的费用依停留时长长短而定:
在德国境内停留超过一年的,费用为110欧元;
在德国境内停留超过3个月不满一年的为80欧元。
\end{itemize}

\section{住宿}
TUM 陆续向学生发邮件询问是否接受校方宿舍,并要求如果接受,须在指定时间内明确答复。

据15级赵家瑞表示,14、15级 Garching 校区的 UCAS 学生都被安排在 Studentenstadt, 离学校三站地铁(约 20min 车程);学校的房子比外面便宜很多;房子可以转手(但网页 \url{https://www.international.tum.de/en/coming-to-tum/exchange-students/accommodation/} 明示:``\textbf{Attention}: Since dormitories of the Munich Student Union are government-funded, it is prohibited to rent this kind of accommodation for personal enrichment. Violation will lead to termination of the rental agreement without previous notice. '')。

\section{航班}
据15级赵家瑞表示,大部分德国人 5 点后不工作;最好下午 2 点到(德国机场?学校?)。

\section{Registration for TUM}
By March 1st, 2019 (only after reveiving access data for TUMonline and TUM regisration number---``Matrikelnummer'')
\begin{enumerate}
\item Transfer the \emph{Student Services Fee} \& \emph{solidary fee for semester ticket}: \euro{129,40} in total
\item German national health insurance (send the proof of insurance called ``Versicherungsbescheinigung'' via email to \href{mailto:alagha@zv.tum.de}{alagha@zv.tum.de} by mentioning your name and your TUM regisration number ``Matrikelnummer''.)
\item Up--load a photo in TUMonline
\end{enumerate}
% NOTE:
% After you logged into TUMonline, a new application form appears. Please DO NOT fill out this online application form in TUMonline!! Please ignore/cancel this step!!!

Following are quoted from a letter from Ms. Alagha:
\begin{quotation}
Access data (link \& username \& password) {\color{gray}(to TUMonline, added by Li Mingyang)}:
\begin{itemize}
\renewcommand\labelitemi{--}  % Following https://tex.stackexchange.com/questions/62496/itemize-with-a-dash-instead-of-a-bullet
\item Will be sent out via e-mail from it-support soon (it-support@tum.de)!
Please check also your spam folder!\\
Please \textbf{wait for it}: also if your friends already might have received the access data, please be patient!
\item \textbf{Password need to be {\color{blue}utilized and changed within a few days upon receipt}, otherwise it will get useless!}
\item if you already registered for TUMonline by mistake, you will not receive new access data. Please login in TUMonline with the same access data you used for the registration.
\end{itemize}
 
\textbf{Important:}\\
After you logged in TUMonline a new application form appears: Please \textbf{DO NOT} fill in this online application form!! Just \textbf{\color{blue}ignore/cancel this step}!!! So please do \textbf{NOT} enter any personal data in TUMonline! Just upload a photo as mentioned below.
\end{quotation}

\section{附录}\label{ap}
% Intentionally add a new page, in convenience for print the following table of contacts.
\newpage

\begin{table}[!htbp]
\centering
\begin{tabular}{ll}
\toprule
Contact in China: & 张顶兰 \\
& Program Coordinator, Study Abroad Office \\
& University of Chinese Academy of Sciences \\
& Room 109, Teaching Building \\
& No.19A, Yuquan Rd, Beijing 100049, China \\
& Tel: +86-10-88256071 \\
& Email: zhangdinglan@ucas.ac.cn \\ \midrule
Contact in Germany: & Dalma Alagha \\
& Student Mobility \\
& Technical University of Munich\\
& TUM International Center \\
& Arcisstr. 21 \\
& 80333 Munich \\
& Phone +49 89 289 23260 \\
& Email: Alagha@zv.tum.de \\ \bottomrule
\end{tabular}
\end{table}

% Intentionally add a new page, in convenience for print the above table of contacts.
\newpage

\begin{table}[htbp]
  \caption{汇款账户(APS 注册成功后会自动生成该表格内容)}
  \label{tb:bank-account}
  \centering
  \begin{tabular}{ll}
    \toprule
    收款人名称 & 德国驻华使馆文化处留德人员审核部 \\
    帐号 & 332 456 013 427 \\
    收款银行名称 & 中国银行北京亮马河大厦支行 \\
    \bottomrule
  \end{tabular}
\end{table}

\begin{center}
\begin{tabular}{ll}
  \toprule
  目的 & 境外银行\textbf{开户} \\ \midrule
  需要证明 & \textbf{资金来源证明(存款证明),\color{blue}中英文},同一张纸上即可 \\ \midrule
  方式 & 在任意一家可办理\textbf{\color{blue}境外汇款业务}的银行,\\
  & (中国银行、建行、工商银行、农行均可) \\
  & 冻结约 9000 \textbf{\color{blue}欧元}(不必过多), \\
  & 并要求银行开具\textbf{资金来源证明}或\textbf{存款证明}。该证明上时间只需持续一天。 \\
  & 如 2018 年 12 月 20 日至 2018 年 12 月 21 日。 \\
  冻结时间 & 约 2 周。(这段时间学生会去申请开户) \\ \midrule
  邮寄 & 将该证明\textbf{原件}邮寄过来。 \\ \midrule
  汇款 & {\color{blue}时间}: 学生办理开户后,通知汇款时间。 \\
  & {\color{blue}金额}: 与开户时填写数目一致(以\textbf{\color{blue}欧元}汇出)。 \\
  & {\color{blue}汇款人}: \textbf{资金证明的账户\color{blue}持有人}! \\
  & {\color{blue}汇款地址}: 见下页表格 \\
  & \textbf{一次性汇出。金额不多不少。} \\
  & 填写汇款人住址时还需写清所在\textbf{城市}。 \\
  & 国内外费用承担建议选择:共同 SHA。 \\
  & 如汇款行要求填写收款人地址,可使用收款人开户银行的地址信息。 \\ \midrule
  \textbf{汇款后} & 在汇出行完成国家外汇管理局要求的\textbf{跨境支出申报}。 \\  
  \bottomrule
\end{tabular}
\end{center}

% \newpage

\begin{table}
\caption{汇款地址信息}
\label{tb:address}
\begin{center}
\begin{tabular}{l}
\toprule
收款人开户银行名称及地址(57): Deutsche Bank AG, Frankfurt H.O. \\
\multicolumn{1}{c}{Taunusanlage 12, 60325 Frankfurt, Germany
 (SWIFT: DEUTDEFFXXX)} \\
收款人名称(59): Deutsche Bank (China) Student BJ \\
收款人账号(59): IBAN: DE64 5007 0010 0951 2724 01 \\
汇款附言(70): 开户人姓名拼音(大写): \underline{\hspace*{3cm}} 个人编号(REFERENCE NUMBER):\\
 $\rule{3cm}{0.15mm}$(共 9 位组成) \\ \bottomrule
\end{tabular}
\end{center}
\end{table}

\newpage

\begin{table}[htbp]
\caption{2018 年 12 月 19 日德意志银行开户体验(及后续同学完善)}
\label{tb:opening-Dec-19}
\centering
\begin{tabular}{l}
\toprule
出发地点:玉泉路 \\
出发时间: 9:02 \\
到达地点:朝阳区建国路 81 号华贸中心 27 层 \\
到达时间: 9:55 \\
取号等待 \\
办理开户 \\ \bottomrule
\end{tabular}
\end{table}
\noindent
备注:
\begin{description}
\item[行程] 地铁一号线大望路站,刷卡(码)出站后,顺着 \textbf{\color{blue}A (东北)口}出站方向,前行约十米,沿着\textbf{\color{blue}“华贸中心”通道}前行。进入“华贸中心”(B1,负一层)后,按指示路牌 \textbf{\color{blue}“华贸写字楼”} 方向前行至一上行扶梯处。出扶梯左手入口即有``Deutsche Bank'' 标志。有服务人员在柜台等候。按指示在机器上选 27 层德意志银行,得到二维码。刷码进入写字中心。乘梯至 27 层,左手即为德意志银行。
\item[表格]
\begin{itemize}
  \item 请务必下载\href{https://china.db.com/china/docs/1.opening\_a\_bank\_account\_for\_foreign\_students\_over18years.pdf}{\textbf{\color{blue}官方最新版表格}}. 填写完整后打印 ``Guidance notes'' 后的所有页(文档当前版本:
  007 91995 28 DBEN 164 WWW ERO BV VJ \textbf{\color{blue}181024} 的\textbf{\color{blue}第 9 -- 19 页}),包括 ``Schufa'' 1 页, ``Entgeltinformation'' 3 页和 ``Depositor Information Sheet'' 1 页。 
  \item 请确保自动生成了 ``\textbf{\color{blue}First name/s}'', ``\textbf{\color{blue}Surname}'' 两栏
  \item 身份证、护照页复印件\textbf{避免过黑}。在空白处写上个人 “姓名” + “拼音”(此前已经有人建议姓名全部大写。我全部大写了)
  \item 按照工作人员的意思,邮寄可能比自取\textbf{\color{blue}晚“一到两天”}。我选了邮寄。邮寄方式应该为 EMS.
  \item 银行张贴的开户示例与官网示例稍有不同---主要是“城市”全部写成“\textbf{\color{blue}省份+城市}”(或城市+省份。可由逗号或空格分隔。下同)。我拍了前两张,将发群里。
  \begin{itemize}
    \item ``Place of birth'' 省份+城市
    \item ``Regestered address'' 中 ``Town/City'' 省份+城市。我填了 ``BeijingShijingshan''. 没有空格,因为格数不够。
    \item ``My relationship with this person is as follows (e.g. father/son)'' 我填写了 ``Mother\textbf{\color{blue}/Son}''. 注意,后面的 ``/Son(/Daughter)'' \textbf{\color{blue}不可省}。
  \end{itemize}
\item 靳慧欣:材料不要订起来
\item 据说华贸大厦马路对面有家打印店,可搜索到。打印费较贵, 0.5 元/张(张兆昱:“那边打印店 2 元一张)。但我高德地图搜不到。仅需打印修改过的页面。{\color{blue}无需打印自己保留的那一份}。我是托旁边人一起打印的,打印店具体位置等不清楚。而且请注意{\color{blue}打印不清晰的,应要求重新打印}。我有几页打印太不清晰。幸好那几页所有人都相同,台湾的一位朋友把他剩余的那几页给了我。
\item 由于去中国银行、工商银行(均在华贸中心一楼。工商银行需转角出门再进。可询问服务台)\textbf{均需\color{blue}当前银行卡},暂不接受现金汇款;而我没有这两个银行的卡,于是我回到玉泉路建行汇出开户手续费。%据赵家瑞表示,德意志银行\textbf{\color{blue}不会通知}是否收到该笔汇款。我托张兆昱明天问问德意志银行。
\\
% 这件事情最好有个彻底的处理方式。比如,要求银行受到汇款后发出通知。
张兆昱:“关于汇款问题,假如你没汇款是会有电话通知。没问题就不会联系。”
\item 银行仅两个柜台。我当天仅一位业务人员。
\item 我全程感觉到的都是{\color{blue}和善}。业务人员也很耐心。本来办公时间只到 11:30, 但她允许我们最迟 12:00 递交材料。
\end{itemize}
\end{description}
\end{document}