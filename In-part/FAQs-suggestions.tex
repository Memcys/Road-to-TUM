\documentclass{ctexart}
\usepackage[official]{eurosym}
\usepackage[colorlinks]{hyperref}
\usepackage{bookmark}
\usepackage{xcolor}

\begin{document}
\section{FAQ}\label{chap:faq}
\begin{enumerate}
  \item 请问需要准备一些欧元现金吗? \\
  (ZJR) 现金 \euro{700},或现金 \euro{$100 \sim 200$} + 信用卡。(第一个月拿不到德意志银行卡。)
  \item 请问(15 级)在德国还用中国手机号吗? \\
  (ZJR) 可以用,不过开通国际漫游很贵。淘宝上可以买德国手机号。去之前装上激活。
  \item 存款证明能以学生本人的身份开具吗? \\
  (ZJR) 不能。德意志银行认定学生本人没有资金收入。
  \item 请问资金冻结只需冻结到开户前一天吗? \\
  (ZJR) 还是冻结到开户后比较靠谱。(LMY) 小地方的银行支行,冻结程序、国际汇款程序可能都不太顺利。
  \item 请问德银开户申请表上,后续汇款金额需要填写大于 0 欧吗? \\
  (ZJR) 需要,不然可能拒绝开户。
  \item 德意志银行会在收到汇款后,通知个人吗? \\
  (ZZY) 未汇款会有电话通知。没问题就不会联系。
  \item 签证需要面签吗? \\
  (ZJR) 不需要。我们只需递签。而且工作人员是中国人。递签时工作人员指着一行德文(AOK 的保险证明没有盖章。需要有类似“未经盖章也生效”的字样),说这写的是什么难道你不知道吗?我递签时表示自己不会德语。检查材料的工作人员没有说我什么。
  \item 递签材料中的保险证明是什么? \\
  (LMY) 是 \textbf{``Versicherungsbescheinigung''}(“保险证明”),\textbf{必须包含 ``Diese Bescheinigung ist maschinell erstellt und auch ohne Unterschrift gültig.''}(“此证书由计算机创建, 并且在没有签名的情况下也是有效的。"){\color{gray}保险起见,可以同时携带 ``Mitgliedschaftsbescheinigung''(“会员证书”),不过该文件不被递签处认可,因为没有 ``Diese Bescheinigung ist maschinell erstellt und auch ohne Unterschrift gültig'' 字样。}
  \item 什么时间都能预约递签吗? \\
  (ZJR) 是。计划好时间后尽快预约。不是每周都能去递签。(LMY) 一周中也可能只有一天能够递签。
  \item 预约递签的网址? \\
  (LMY) \href{https://service2.diplo.de/rktermin/extern/choose_category.do?locationCode=peki&realmId=12&categoryId=156&request_locale=de}{在线系统预约签证受理}。见 \ref{sec:visa} 节。请务必选择德语界面。英语界面会有部分文字缺失。德语界面在填写时会提供 德/英/中 三语。
  \item 机票提前买有优惠吗? \\
  (LMY) 我们在 12 月、1 月没有看到机票价格的明显上涨(涨幅似乎不超过 100 元人民币)。
  \item 德银开户时,可以只开半年限制性提款账户吗? \\
  (LHB) 大使馆回信(见 \ref{fig:email}),可以。也即只需 \euro{4320}。
  \item 购买去慕尼黑的机票时,需要同时购买返程机票吗? \\
  (ZJR) 不需要。我们是长期签证。
  \item 在德银办理的一年期限制性提款账户,半年回国后,剩下的钱怎么办? \\
  (ZJR) 回国前可以用机票和学校的注销证明去德意志银行解冻。
  \item 翻译成德文的在学证明需要盖章吗? \\
  不需要。
  \item 签证是否和护照一同寄出? \\
  (LMY) 如果申请签证成功,签证将粘贴在护照中,一同寄出;否则,仅寄回护照原件。
  \item 签证通过与否都会收到邮件通知吗? \\
  (PZY) 我记得是拒签了才会邮件通知吧。
  \item 什么时间开始选课? \\
  (LMY) 没有统一时间。请关注 \href{campus.tum.de}{TUMonline} 上的课程列表。
  \item 开封过的日用品可以过海关吗? \\
  (ZJR) 视其种类而定。(待完善)
  \item 有关宿舍的分配,为什么还没有收到邮件? \\
  (LMY) 我们在 3 月上旬陆续收到宿舍(再次)确认的邮件。请耐心等待。
  \item 申请住宿的表格需要打印并手写,拍照发回吗? \\
  (ZJR) 可以。也可以电子签名。(LMY) 请按要求发送 .pdf 格式文件。
  \item 慕尼黑学联的接机服务,我们作为访学生可以报名吗? \\
  (ZJR) 可以。
  \item 慕尼黑宿舍网络情况如何? \\
  (ZJR) 入住时缴网费,\euro{20}一学期,网速还行。建议自带三米长网线。需要路由器的也自带(听说那边路由器信号不好)。

  (以下来自邮件 ``Wichtige Informationen für Ihr Austauschstudium an der TUM /important information for your exchange at TUM'' 中附件)
  \item What to do for enrollment, and until when? \\
  Pay your fee, send proof of German national health insurance, and upload a photo
  (see TUM admission letter), latest by March 1 st , 2019!
  \item When and how do I know that I’m enrolled at TUM? \\
  When: from mid-March – after TUM finalized the enrollment process for all exchange students. \\
  How do I know it: you will receive an automatic e-mail from TUM IT-support.
  \item When will I receive my TUM e-mail address and what is it for? \\
  When: from mid-March. \\
  What is it for: register for courses and exams, receive important information, download documents. \\
  What else do I need to do: Please do not forget to forward your e-mails from your TUM-account to your
  private e-mail account (or the other way around). The International Center sends mails only to your private
  e-mail address. TUM professors and the admission office send e-mails to your TUM e-mail address. Further
  information in this regard can be found \href{http://www.it.tum.de/en/faq/it-services/e-mail/}{here}.
\end{enumerate}

\section{一些建议}
\section{德银开户建议}
\begin{itemize}
  \item 在学证明一次打印盖章即可,共 5 份(我只打了 4 份)。
  \item 大学录取花名册按录取省份划分。建议同一省份同学帮本省同学代领。(但我们没有尝试过)
  \item (ZJR) 在德意志银行申请开户时,一定要带上电脑、U 盘,随时准备重新填表。(见 \ref{de-bank-ZJR-laptop-USB}。)
  \item (ZJR) 字母大小写的问题,一般来说,大写肯定不会出问题。(后面的签证材料也同)
  \item (JHX) 德银开户处交待,不要把材料订起来。
\end{itemize}

\section{递签建议}
\begin{itemize}
% \item (ZZY) 强烈建议所有材料备份两份。
% \item (PZY) 递签材料中,保险声明、联系人声明、申请表空白表可以多打印几份。
\item (PZY) 递签处禁止手机(等电子设备)。可以带纸质书去消遣。
\item (ZZY, PZY) \textbf{\color{blue}空白表最好多备几份},比如联系人声明,保险声明。
\item\label{print-fee} 照片也多备几张。
\item 手写内容可在场填写。不确定的地方,可以询问周围学生或保安。
\item\label{ensurance-Versicherungsbescheinigung} (PZY) 我们此次出问题的主要是\underline{保险证明}---\textbf{\color{red}务必打印 ``Versicherungsbescheinigung'' 表格},因为需要其中的 ``Diese Bescheinigung ist maschinell erstellt und \textbf{\color{red}auch ohne Unterschrift gültig}'', 主要是需要其中的“在没有签名的情况下有效”的声明(这也是所有德国未加盖章的文档所需的)---和\underline{申请表}(内容填写与 ZJR 的有别)%,保险证明最好把两个文件都打印了,申请表可以对照群里新更新的模板补充一下(有空更新)。 
\item EMS 快递单可在 APS 取号后,由一人前往办理(需填写邮寄地址、邮编、手机号及姓名)。
\item (ZJR) 签证材料一定要按照要求准备。例如,照片上刘海稍许遮住眉毛,差点被打回重新递签,幸亏有备份照片。材料中信息要一致,如出境日期,所有文件都要填同一日期。
\item (ZJR) VIDEX 表格中停留日期填写 90 天,从 4 月 1 日开始。
\item (ZJR) 记得有一句,不是英语的,需提供英语翻译。在学证明需要中英文版本的(附加德文翻译)。
\end{itemize}
%

\section{其他建议}
\begin{itemize}
  \item 访学离校需办理“\href{http://bkjy.ucas.ac.cn/index.php/fxjl/download/3401-2017-06-13-03-47-14?task=down&fid=818}{离校手续单}”和学生、家长签字的“\href{http://bkjy.ucas.ac.cn/index.php/fxjl/download/3400-2017-06-13-03-46-26?task=down&fid=701}{公派赴境外学习交流责任书}”。建议寒假期间打印并填写好。
  \item 所有公证件和翻译件的原件、复印件一定要带多份!不然你到了德国再去准备,真的非常麻烦。(来自 \url{https://zhuanlan.zhihu.com/p/57766122})
  \item 准备在德国可用的(导航)地图(软件等)。
  \item (PZY) 网页推荐: \href{https://mp.weixin.qq.com/s?srcid=0319fCMV7x6jt2w4xgPGYSMm&scene=23&mid=2651269086&sn=2b9eff5e72b2e989ac404be9be21bc09&idx=2&__biz=MzA4NTM2MzAxNw%3D%3D&chksm=842a97f8b35d1eee47778e42c10d883071d1546d6bab41132685fd4c983a6d713ca70a9b2328&mpshare=1#rd&appinstall=0}{在德国落地之后最要紧的7件事!}
\end{itemize}
\end{document}